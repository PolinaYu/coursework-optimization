\chapter*{Введение}
\addcontentsline {toc}{chapter}{ВВЕДЕНИЕ}

Проблема распространения эпидемии всегда была важным вопросом в обществе. Особенно в прошлом человечество много страдало из-за очень заразных болезней. Одним из наиболее значительных примеров является Черная Смерть, пандемия бубонной чумы, которая унесла жизни 1/4 100-миллионного населения Европы в 14 веке. Описания этих событий мы можем найти, например, в шестой книге Лукреция  “О природе вещей” (“De Rerum Natura”). Математический подход к этой проблеме также имеет довольно долгую историю, вероятнее всего, инициированную Д. Бернулли в 1760 году \cite{Anderson-May}. В 20-м веке эта проблема интенсивно изучалась: ученые свели эту чрезвычайно сложную проблему к наборам дифференциальных уравнений и нескольким параметрам, таким как скорость контакта между восприимчивыми и инфекционными людьми, а также переменным, таким как плотность восприимчивых людей и т. Д. Широкий обзор литературы, связанной с классическим подходом к математическому моделированию, можно найти, например, в работах \cite{Anderson-May},\cite{Bailey}.\newline
Основные вопросы математического моделирования распространения эпидемии тесно связаны с теми, которые мы задаем при решении проблем общественного здравоохранения. То есть: в каких условиях можно ожидать вспышки эпидемии? Сколько людей будет заражено? Сколько времени требуется, чтобы стабилизировать ситуацию? На все эти вопросы чрезвычайно трудно ответить в реальности. Существует так много факторов, влияющих на распространение эпидемии: степень заражения конкретным вирусом/бактерией,частота и характер контактов между людьми, продолжительность заболевания, устойчивость или иммунитет конкретных людей или, например, погода. Именно поэтому в решении эпидемиологических проблем участвуют исследователи из многих дисциплин, таких как биология, компьютерные науки, социальные науки, физика и, что не менее важно, математика. \newline
С момента вспышки коронавируса COVID-19 в начале 2020 года вирус поразил большинство стран и унес жизни более 600 тысяч людей по всему миру. К марту 2020 года Всемирная организация здравоохранения (ВОЗ) объявила ситуацию пандемической, первой в своем роде в нашем поколении. На сегодняшний день многие страны и регионы были заблокированы и применяли жесткие меры социального дистанцирования, чтобы остановить распространение вируса. С точки зрения стратегии и управления здравоохранением, модель распространения заболевания и прогнозирование его распространения во времени имеют большое значение для спасения жизней и минимизации социальных и экономических последствий заболевания.


\bigskip

Еще раз подчеркнуть цель работы (не повторять указанную в реферате).

Кратко изложить содержание работы, примерно в таком виде: В частности, в разд. 1 обосновано …. В разд. 2 исследуется ….. В разд. 3 продолжается исследование задач …. В разд. 4 эффективность предложенных методов иллюстрируется численными примерами… В заключении приводятся краткие выводы по результатам проведенной работы и даются рекомендации о перспективах дальнейших исследований по исследуемой тематике.
